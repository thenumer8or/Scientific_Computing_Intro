\documentclass[11pt]{article}
\usepackage[margin=1.0in]{geometry}
\begin{document}
\title{Third programming Assignment}
\author{Peter Monk}
\date{\today}
\maketitle

You may work in pairs on this assignment (or alone if you prefer).  Please continue to refer to the Navier-Stokes notes from Sakai.  Your job is to implement the non-linear convective term to convert your Stokes code to a Navier-Stokes code.   So if $\vec{u}=(u,v)^T$ we will now be be solving the Stokes system 
\begin{eqnarray*}
\frac{\partial \vec{u}}{\partial t}+(\vec{u}\cdot\nabla)\vec{u}+\nabla p&=&\frac{1}{Re}\Delta\vec{u}\\
\nabla\cdot\vec{u}&=&0
\end{eqnarray*}
on a simple unit square so $\Omega=[0,1]\times [0,1]$.  In addition we will impose the following, somewhat artificial, ``driven cavity'' boundary conditions
\begin{eqnarray*}
u=v=0\mbox{ when }x=0\\
u=v=0\mbox{ when }x=1\\
u=v=0\mbox{ when }y=0\\
u=\bar{u},\;v=0\mbox{ when } y=1
\end{eqnarray*}
where $\bar{u}$ is a given constant and is specified as \verb+ubar+ in my code.

Modify your function
\begin{verbatim}
 function   [f,g]=comp_fg ( u, v, imax, jmax, delt, delx, dely, re )
    \end{verbatim}
    to include the convective term using equations () and () from the notes (implementing upwind differencing for stablity).

Implement a new function that computes the maximum stable time step using equations ().  Make the first line of this function
\begin{verbatim}

\end{verbatim}
This will need to be used at the start of every time step (i.e. you can no-longer use a fixed time step) so you will need to slightly change the time step stopping criterion.  You can no longer expect to hit $t=5$ exactly, but run until you have just passed that time.



\end{document}